\documentclass[12pt]{article}
\usepackage[paper=letterpaper,margin=2cm]{geometry}
\usepackage{amsmath}
\usepackage{amssymb}
\usepackage{amsfonts}
\usepackage{newtxtext, newtxmath}
\usepackage{enumitem}
\usepackage{titling}
\usepackage[colorlinks=true]{hyperref}
\usepackage[spanish]{babel}

\setlength{\droptitle}{-6em}

% Enter the specific assignment number and topic of that assignment below, and replace "Your Name" with your actual name.
\title{Práctica 2 - Programación Concurrente y de Tiempo Real}
\author{Víctor Moreno Sola}
\date{\today}

\begin{document}
\maketitle

\textbf{Ejercicio 1 - hebra.java y Usa\_hebra.java}
    \begin{table}[h!]
        \centering
        \begin{tabular}{c|c}
         Nº de iteraciones & Valor de n \\
         \hline
         1000 & 0 \\
         10000 & 76 \\
         100000 & 57808 \\
        \end{tabular}
        \caption{Analísis}
        \label{tab:Tabla 1}
    \end{table}
    
    Como podemos observar en Cuadro 1, a mayor número de iteraciones mayor es el valor de n.

\end{document}
